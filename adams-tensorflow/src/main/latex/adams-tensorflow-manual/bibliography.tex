% This work is made available under the terms of the
% Creative Commons Attribution-ShareAlike 4.0 license,
% http://creativecommons.org/licenses/by-sa/4.0/.

\begin{thebibliography}{999}
	% to make the bibliography appear in the TOC
	\addcontentsline{toc}{chapter}{Bibliography}

    % references
	\bibitem{adams}
		\textit{ADAMS} -- Advanced Data mining and Machine learning System \\
		\url{https://adams.cms.waikato.ac.nz/}{}
		
	\bibitem{tensorflow}
                \textit{TensorFlow} -- an open source machine learning framework for everyone. \\
		\url{https://www.tensorflow.org/}{}

	\bibitem{objdet}
                \textit{Object Detection API} -- The TensorFlow Object Detection API is an open
                source framework built on top of TensorFlow that makes it easy to construct,
                train and deploy object detection models. \\
		\url{https://github.com/tensorflow/models/tree/master/research/object_detection}{}

	\bibitem{tfrecord}
                \textit{tfrecord} -- The TFRecord format is a simple format for storing a
                sequence of binary records. \\
		\url{https://www.tensorflow.org/tutorials/load_data/tfrecord}{}

	\bibitem{waitfrecords}
                \textit{wai.tfrecords} -- A Python library to generate TFRecord files (based on
                Object Detection API code). \\
		\url{https://github.com/waikato-datamining/tensorflow/tree/master/tfrecords}{}

\end{thebibliography}
